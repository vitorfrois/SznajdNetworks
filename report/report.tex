\documentclass{article}

% Language setting
% Replace `english' with e.g. `spanish' to change the document language
\usepackage[english]{babel}

% Set page size and margins
% Replace `letterpaper' with `a4paper' for UK/EU standard size
\usepackage[letterpaper,top=2cm,bottom=2cm,left=3cm,right=3cm,marginparwidth=1.75cm]{geometry}

% Useful packages
\usepackage{amsmath}
\usepackage{graphicx}
\usepackage[colorlinks=true, allcolors=blue]{hyperref}

\title{Predição de Variáveis Dinâmicas no Modelo de Sznajd em Redes Complexas}
\author{Vítor Amorim Fróis}

\begin{document}
\maketitle

\begin{abstract}
    O presente projeto utiliza Aprendizado de Máquina para prever variáveis complexas no modelo de Sznajd em Redes Complexas: o Tempo de Consenso e a Frequência de Troca de Opinião. Ao utilizar medidas topológicas para caracterização de redes e consequentemente como features, podemos prever as variáveis com alta acurácia.
    Ao explorar a convergência entre estrutura e dinâmica de redes, esse projeto responde dúvidas relacionadas aos mecanismos de polarização em interações sociais.
\end{abstract}

\section{Introduction}
A interação entre componentes de um sistema que possuem regras simples leva a formação de padrões complexos e características como emergência, livre de escala e heterogeneidade. Fenômenos emergentes são presentes em sistemas complexos e caracterizados pelo resultado espontâneo da interação entre os milhares de componentes que constituem o sistema. Um grande exemplo de emergência ocorre durante a noite do sudeste asiático, quando vagalumes da região piscam de acordo ajustam a frequência do piscar de suas luzes de acordo com os vizinhos mais próximos, até que o efeito seja extendido por todo o sistema, de forma que os indivíduos pisquem em sincronia \cite{johnson2002emergence}.

No contexto de dinâmicas sociais, isto é, modelos matemáticos que buscam reproduzir o comportamento humano em redes, a emergência pode ser caracterizada como um fenômeno relacionado a polarização \cite{maia2021emergence}. Aqui e no restante do relatório, definimos polarização como a fragmentação de opiniões, um estado contrário ao consenso. Diversos estudos mostram que a polarização pode ter profunda influência no âmbito político \cite{interian2023polarization,layton2021polarization}. Dessa forma, o estudo das causas da polarização é de suma importância para mitigar os impactos causados pelo seu desenvolvimento.

A física estatística desenvolveu ferramentas para o estudo de sistemas de muitas partículas interagentes, os quais são adaptados com facilidade para o estudo de dinâmicas sociais. Ersnt Ising encontrou a solução exata para um modelo de paramagneto, representando materiais que podem alcançar dois estados conflitantes e buscam um estado de mínima energia. O modelo recebeu o nome de Ising e pode ser considerado como um modelo para simples opiniões, onde há uma transição de fase entre os estados de polarização e consenso. O modelo de Sznajd foi inspirado pelo primeiro modelo e busca explorar como opiniões semelhantes são necessárias para influenciar outros. Já o modelo votante ilustra como a maioria pode influenciar vizinhos, explorando por sua vez como a ordem emerge a partir da opinião maioria.

\begin{itemize}
    \item Sistemas complexos
    \item Polarização e motivação política
    \item ising $\rightarrow$ sznajd, voter e q-voter
    \item topologia pode influenciar na formação de consenso, além disso seria ótimo estudar sistemas complexos com ferramentas como ML
    \item alguns resultados com o modelo de sznajd
    \item vamos comparar diferentes abordagens e dinâmicas
\end{itemize}

\section{Materiais e Métodos}
\subsection{Geração de Redes Aleatórias}

\subsection{Simulação de Monte Carlo do modelo de Sznajd}
\subsection{Medidas}
\subsection{Machine Learning}
selecao empirica de features, selecao via forward selection, r2 score, regressao nao linear
\section{Resultados}
Falar brevemente sobre a alta acurácia em cada um dos casos (random, direto e inverso)
\subsection{tabela com os resultados}
\subsection{medidas mais importantes}
\section{Conclusão}
\begin{itemize}
    \item{pq as medidas mais importantes sao importantes}
    \item{quais as diferencas entre as inicializacoes diferentes}
    \item{regressao poisson vs xgboost e redes neurais}
    \item{Comparacao q-voter}
\end{itemize}

\bibliographystyle{alpha}
\bibliography{sample}

\end{document}